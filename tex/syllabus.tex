\setcounter{chapter}{-1}
\chapter{Syllabus}
\label{cha:syllabus}
\setcounter{page}{1}
\pagestyle{fancy}

\fcolorbox{gray!25}{gray!25}{
    \centering
    \begin{tabular}{ll}
        \textbf{Course:} Parsing and Syntactic Processing \quad\qquad\qquad&
        \textbf{Name:} Thomas Graf\\
        \textbf{Course\#:} Lin630 (officially Lin650) &
        \textbf{Email:} lin630@thomasgraf.net\\
        \textbf{Time:} M 3:00--3:50, W 2:00--3:45&
        \textbf{Office hours:} M11--12, F11--1, N249\\
        \textbf{Course Website:} on Blackboard &
        \textbf{Personal Website:} http://thomasgraf.net\\
    \end{tabular}
}

\section{Overview}
\begin{itemize}
    \item \textbf{Big Questions}
        \begin{itemize}
            \item What is the relation between competence and performance, grammar and parser?
            \item Are syntactic processing effects conditioned by the grammar?
            \item What qualifies as a parser as opposed to a recognizer or a parsing procedure?
        \end{itemize}
        The first two are common questions for any processing course.
        The third one hints at the special twist of this course: we approach these issues from a computational perspective!
        Parsing theory is a big (albeit messy) area of computer science, there's tons of parsing models on the market.
        So let's bring all these insights to bear on how humans parse natural language.

    \item \textbf{Teaching Goals}\\
        At the end of this course you will 
        \begin{itemize}
            \item be familiar with a variety of common parsing models (top-down, bottom-up, left-corner, Earley, CYK)
            \item know the most common syntactic processing effects (in particular those related to memory usage)
            \item be able evaluate claims in the psycholinguistic literature from a computational perspective
            \item possibly have a paper ready to publish (there's some low-hanging fruits here)
        \end{itemize}

    \item \textbf{Prerequisites}\\
    None beyond basic syntax skills --- you should be able to draw a reasonable tree for a sentence like \emph{The fact that the employee who the manager hired stole office supplies did not go unnoticed by the janitor}.
\end{itemize}

\section{Course Requirements}
\begin{itemize}
    \item\textbf{Readings}
        \begin{itemize}
            \item Read every assigned paper (there won't be that many)
        \end{itemize}
        %
    \item \textbf{Homeworks}
        \begin{itemize}
            \item Short homeworks for each parsing model
            \item Participate in discussion of selected homework exercises
        \end{itemize}
        %
    \item \textbf{Paper (3 credits only)}\\
        The second half of this course is meant to be run like a research group where we play around with current parsing models of Minimalist grammars.
        Depending on how things develop, we might get a joint paper out of this, or maybe you'll work on something on your own.
        At any rate you must hand in a paper of at least 8 pages with your name on it by the end of the course.
\end{itemize}

\section{Outline}

\begin{longtable}{rlp{10cm}}
    \emph{Wk} & \emph{Date} & \emph{Topic}\\\hline
    1 & Aug 25, 27 & Marr's levels, parser VS grammar, phrase structure grammars\\
    2 & Sep 3 & what is a parser (technical answer)\\
    3 & Sep 8, 10 & top-down parsing\\
    4 & Sep 15, 17 & bottom-up parsing\\
    5 & Sep 22, 24 & left-corner parsing\\
    6 & Sep 29, Oct 1 & Earley \& CKY\\
    7 & Oct 6, 8 & everything our parsers miss\\
    8 & Oct 13, 15 &  why the grammars aren't right\\
    9 & Oct 27, 29 & Moving to Minimalist grammars\\
    10 & Nov 3, 5 & Minimalist top-down parsing of derivation trees\\
    11 & Nov 10, 12 & More on MG parsing\\
    $\geq$12 & & research meetings
\end{longtable}
% chapter syllabus (end)
