\chapter{A Move-Eager Parser for MGs}
\label{cha:StablerParser}

The context-free parser for MGs has two disadvantages.
First, it freely infers feature tuples, with the only restriction being that the features of these tuples are features of the grammar.
Consequently, the parser may conjecture tuples that never occur in a well-formed derivation of the grammar, which is a major waste of resources.
Second, its search through the tree is not contingent on conjectured landing sites.
The parser follows a simple pattern of ``specifiers before complements'' pattern, with the minor twist that if a mover is found, the parser inserts it at the final target site and continues parsing from there.
This means that the parser has to store a lot of material in memory.
If it progressed on the most direct path to where it expects the mover to originate, without building any of the structure along the other paths, it may safe quite some resources.
Let's call the first type of parser \emph{Merge-eager}, and the second \emph{Move-eager}.
The top-down MG parser defined in \citet{Stabler12} is Move-eager and operates directly on the lexicon, which prevents it from conjecturing useless feature tuples.

\section{A Closer Look at the Stabler Parser}
See the paper for details, in particular the appendix.
The crucial insights are as follows:
%
\begin{itemize}
    \item Every Minimalist lexicon can be represented as a \emph{prefix tree}, where the prefixes correspond to the feature components of lexical items read from right to left.
    \item The prefix tree guides the parser in the structure building process.
    \item In order to deal with non-determinism, the parser keeps multiple parses in memory.
    \item Parses are ordered by their probability, and those with a probability below a fixed threshold $p$ are discarded.
        This is called \emph{beam parsing}.
    \item Since the probability of a tree decreases with every level of embedding, left branch recursion is no longer a problem for the parser.
        Instead of adding more and more levels of embedding \emph{ad infinitum}, the parser eventually reaches the probability threshold $p$ and conjectures no further levels.
\end{itemize}

\section{Psycholinguistic Adequacy}
\subsection{Generalizations about Relative Clauses}
Two major properties of relative clauses are firmly established in the literature (see \citealt{Gibson98} and references therein).
%
\begin{itemize}
    \item \textbf{SC\slash RC $<$ RC\slash SC}\\
        A sentential complement containing a relative clause is easier to process than a relative clause containing a sentential complement.
    \item \textbf{SubjRC $<$ ObjRC}\\
        A relative clause containing a subject gap is easier to parse than a relative clause containing an object gap.
\end{itemize}
%
These generalizations were obtained via self-paced reading experiments and ERP studies with minimal pairs such as \eqref{ex:StablerParser_SC/RC-RC/SC} and \eqref{ex:StablerParser_SubjRC-ObjRC}, respectively.
%
\begin{exe}
    \ex\label{ex:StablerParser_SC/RC-RC/SC}
    \begin{xlist}
        \ex The fact [\textsubscript{SC} that the employee$_i$ [\textsubscript{RC} who the manager hired $t_i$] stole office supplies] worried the executive.
        \label{ex:StablerParser_SC/RC}
        %
        \ex The executive$_i$ [\textsubscript{RC} who the fact [\textsubscript{SC} that the employee stole offices supplies] worried $t_i$] hired the manager.
        \label{ex:StablerParser_RC/SC}
    \end{xlist}
    %
    \ex\label{ex:StablerParser_SubjRC-ObjRC}
    \begin{xlist}
        \ex The reporter$_i$ [\textsubscript{RC} who $t_i$ attacked the senator] admitted the error.
        \label{ex:StablerParser_SubjRC}
        \ex The reporter$_i$ [\textsubscript{RC} who the senator attacked $t_i$] admitted the error.
        \label{ex:StablerParser_ObjRC}
    \end{xlist}
\end{exe}
%
We can use these sentences as test cases to determine whether the two types of MG parsers --- Merge-eager and Move-eager --- can account for these discrepancies, and if so, which one of them.

\subsection{Refining our Metrics}
In previous evaluations of parser predictions we used three metrics: payload for the number of nodes kept in memory, MaxTen for the maximum number of steps a node is kept in memory, and SumTen as the sum of all non-trivial tenure.
For the MG-parser, it makes sense to introduce further subtypes of these.
Since MGs have empty heads, the linguistic status of which is contentious, each metric can have two values, the standard one computed over all nodes, and a more restricted one that ignores all nodes that are empty heads.
For the sake of succinctness we write these values in the slashed format $m/n$, where $m$ is the standard value and $n$ the restricted one.
Another distinction that might be useful is the one between lexical items and phrasal nodes, which is why each metric also has a subtype that only considers leafs in the derivation.
These subtypes are indicated by a subscripted \emph{Lex}.
Overall, then, we have four metrics and each metric has a slashed value.
%
\begin{description}
    \item[\BoxTen] number of all\slash pronounced nodes kept in memory for at least 3 steps (this is increased from our previous default of 2, following \citealt{GrafMarcinek14CMCL})
        %
    \item[\BoxLex] number of all\slash pronounced leaves kept in memory for at least 3 steps
        %
    \item[\MaxTen] greatest number of steps that any\slash some pronounced node is kept in memory
    \item[\MaxLex] greatest number of steps that any\slash some pronounced leaf is kept in memory
\end{description}
%
These metrics are taken from \citet{GrafMarcinek14CMCL}, where they are called \textbf{Box},\textbf{BoxLex}, \textbf{Max}, and \textbf{MaxLex}, respectively.

\subsection{Sentential Complements and Relative Clauses}
The Merge-eager and Move-eager derivation trees for \eqref{ex:StablerParser_SC/RC} and \eqref{ex:StablerParser_RC/SC} are given in Fig.~\ref{fig:StablerParser_DerSC-RC_Merge-Eager}--\ref{fig:StablerParser_Der-RC-SC_Move-Eager}.
For the sake of clarity lexical items are given without their features, movement is indicated by dashed branches, and interior nodes are labeled in the fashion of X$'$-theory.

% SC/RC
\begin{figure}[tbph]
\TreeSize
\centering
%
\begin{tikzpicture}[
    scale=\TreeScale,
    level 1+/.style={level distance=\TreeDistanceValue, sibling distance=-.5em},
    level 1/.style={sibling distance=-2em},
    level 4/.style={sibling distance=-25em},
    level 9/.style={sibling distance=-5em},
    level 10/.style={sibling distance=-18em},
    ]
\Tree
[.\Lab{CP}{0}{1}
	[.\Lab{C}{1}{2} ]
	[.\node(TP1){\Lab{TP}{1}{3}};
		[.\Lab{T$'$}{3}{4}
			[.\BLab{T}{4}{36} ]
			[.\Lab{VP}{4}{5}
				[.\node(TP1Subj){\Lab{DP}{5}{6}};
					[.\Lab{the}{6}{7} ]
					[.\Lab{NP}{6}{8}
						[.\Lab{fact}{8}{9} ] 
						[.\Lab{CP}{8}{10}
							[.\Lab{that}{10}{11} ]
                            [.\node(TP2){\Lab{TP}{10}{12}};
                                [.\Lab{T$'$}{12}{13}
                                    [.\BLab{T}{13}{32} ]
                                    [.\Lab{VP}{13}{14}
                                        [.\node(TP2Subj){\Lab{DP}{14}{15}};
                                            [.\Lab{the}{15}{16} ]
                                            [.\node(RCtarget){\Lab{NP}{15}{17}};
                                                [.\Lab{N$'$}{17}{18}
                                                    [.\BLab{who}{18}{27} ]
                                                    [.\node(TP3){\Lab{TP}{18}{19}};
                                                        [.\Lab{T$'$}{19}{20}
                                                            [.\BLab{T}{20}{30} ]
                                                            [.\Lab{VP}{20}{21}
                                                                [.\node(TP3Subj){\Lab{DP}{21}{22}};
                                                                    [.\BLab{the}{22}{28} ]
                                                                    [.\BLab{manager}{22}{29} ]
                                                                ]
                                                                [.\Lab{V$'$}{21}{23}
                                                                    [.\BLab{hired}{23}{31} ]
                                                                    [.\node(RCsource){\Lab{DP}{23}{24}};
                                                                        [.\Lab{D}{24}{25} ]
                                                                        [.\Lab{employee}{24}{26} ]
                                                                    ]
                                                                ]
                                                            ]	
                                                        ]												
                                                    ]
                                                ]
                                            ]
                                        ]
                                        [.\IBLab{V$'$}{14}{33}
                                            [.\Lab{stole}{33}{34} ]
                                            [.\Lab{office supplies}{33}{35} ]
                                        ]
                                    ]
                                ]
                            ]
						]
					]
				]
				[.\IBLab{V$'$}{5}{37}
					[.\Lab{worried}{37}{38} ]
                    [.\Lab{DP}{37}{39}
                        [.\Lab{the}{39}{40} ]
                        [.\Lab{executive}{39}{41} ]
                    ]
				]
			]
		]
	]
]

\draw[dashed,bend left] (TP1Subj) to (TP1);
\draw[dashed,bend left] (TP2Subj) to (TP2);
\draw[dashed,bend left=60] (TP3Subj) to (TP3);
\draw[dashed,bend right=30] (RCsource.north) to (RCtarget);
\end{tikzpicture}
\caption{Merge-eager parse of sentential complement with embedded relative clause;
    \MaxTen$=32/32$,
    \MaxLex$=32/9$,
    \BoxTen$=9/6$,
    \BoxLex$=7/4$}
\label{fig:StablerParser_DerSC-RC_Merge-Eager}
\end{figure}

% RC/SC
\begin{figure}[tbph]
\TreeSize
\centering
%
\begin{tikzpicture}[
    scale=\TreeScale,
    level 1+/.style={level distance=\TreeDistanceValue},
    level 1/.style={sibling distance=-5em},
    level 2+/.style={sibling distance=-.5em},
    level 3/.style={sibling distance=-10em},
    level 4/.style={sibling distance=-20em},
    level 9/.style={sibling distance=-10em},
    level 10/.style={sibling distance=-20em}
    ]
\Tree
[.\Lab{CP}{0}{1}
	[.\Lab{C}{1}{2} ]
	[.\node(TP1){\Lab{TP}{1}{3}};
		[.\Lab{T$'$}{3}{4}
			[.\BLab{T}{4}{37} ]
			[.\Lab{VP}{4}{5}
				[.\node(TP1Subj){\Lab{DP}{5}{6}};
					[.\Lab{the}{6}{7} ]
					[.\node(RCtarget){\Lab{NP}{6}{8}};
						[.\Lab{N$'$}{8}{9}
							[.\BLab{who}{9}{26} ]
							[.\node(TP2){\Lab{TP}{9}{10}};
								[.\Lab{T$'$}{10}{11}
									[.\BLab{T}{11}{35} ]
									[.\Lab{VP}{11}{12}
										[.\node(TP2Subj){\Lab{DP}{12}{13}};
											[.\BLab{the}{14}{27} ]
											[.\Lab{NP}{14}{15}
												[.\BLab{fact}{15}{28} ]
												[.\Lab{CP}{15}{16}
													[.\BLab{that}{16}{29} ]
													[.\node(TP3){\Lab{TP}{16}{17}};
														[.\Lab{T$'$}{17}{18}
															[.\BLab{T}{18}{32} ]
															[.\Lab{VP}{18}{19}
                                                                [.\node(TP3Subj){\Lab{DP}{19}{20}};
																	[.\BLab{the}{20}{30} ]
																	[.\BLab{employee}{20}{31} ]
																]
																[.\Lab{VP}{19}{21}
																	[.\BLab{stole}{21}{33} ]
																	[.\BLab{office supplies}{21}{34} ]
																]
															]
														]
													]
												]											
											]				
										]
										[.\IBLab{V$'$}{12}{22}
											[.\BLab{worried}{22}{36} ]
                                            [.\node(RCsource){\Lab{DP}{22}{23}};
                                                \Lab{D}{23}{24}
                                                \Lab{executive}{23}{25}
                                            ]
										]
									]
								]
							]
						]
					]
				]
				[.\IBLab{V$'$}{5}{38}
					[.\Lab{hired}{38}{39} ]
                    [.\Lab{DP}{38}{40}
                        [.\Lab{the}{40}{41} ]
                        [.\Lab{manager}{40}{41} ]
                    ]
				]
			]
		]
	]
]

\draw[dashed,bend left] (TP1Subj) to (TP1);
\draw[dashed,bend left] (TP2Subj) to (TP2);
\draw[dashed,bend left=60] (TP3Subj) to (TP3);
\draw[dashed,bend right] (RCsource.north) to (RCtarget);
\end{tikzpicture}
\caption{Merge-eager parse of relative clause containing a sentential complement;
    \MaxTen$=33/33$,
    \MaxLex$=33/17$,
    \BoxTen$=14/11$,
    \BoxLex$=12/9$}
\label{fig:StablerParser_Der-RC-SC_Merge-Eager}
\end{figure}

% SC/RC
\begin{figure}[tbph]
\TreeSize
\centering
%
\begin{tikzpicture}[
    scale=\TreeScale,
    level 1+/.style={level distance=\TreeDistanceValue, sibling distance=-.5em},
    level 1/.style={sibling distance=-2em},
    level 4/.style={sibling distance=-25em},
    level 9/.style={sibling distance=-5em},
    level 10/.style={sibling distance=-18em},
    ]
\Tree
[.\Lab{CP}{0}{1}
	[.\Lab{C}{1}{2} ]
	[.\node(TP1){\Lab{TP}{1}{3}};
		[.\Lab{T$'$}{3}{4}
			[.\BLab{T}{4}{35} ]
			[.\Lab{VP}{4}{5}
				[.\node(TP1Subj){\Lab{DP}{5}{6}};
					[.\Lab{the}{6}{7} ]
					[.\Lab{NP}{6}{8}
						[.\Lab{fact}{8}{9} ] 
						[.\Lab{CP}{8}{10}
							[.\Lab{that}{10}{11} ]
                            [.\node(TP2){\Lab{TP}{10}{12}};
                                [.\Lab{T$'$}{12}{13}
                                    [.\BLab{T}{13}{32} ]
                                    [.\Lab{VP}{13}{14}
                                        [.\node(TP2Subj){\Lab{DP}{14}{15}};
                                            [.\Lab{the}{15}{16} ]
                                            [.\node(RCtarget){\Lab{NP}{15}{17}};
                                                [.\Lab{N$'$}{17}{18}
                                                    [.\BLab{who}{18}{26} ]
                                                    [.\node(TP3){\Lab{TP}{18}{19}};
                                                        [.\Lab{T$'$}{19}{20}
                                                            [.\BLab{T}{20}{30} ]
                                                            [.\Lab{VP}{20}{21}
                                                                [.\node(TP3Subj){\IBLab{DP}{21}{27}};
                                                                    [.\Lab{the}{27}{28} ]
                                                                    [.\Lab{manager}{27}{29} ]
                                                                ]
                                                                [.\Lab{V$'$}{21}{22}
                                                                    [.\BLab{hired}{22}{31} ]
                                                                    [.\node(RCsource){\Lab{DP}{22}{23}};
                                                                        [.\Lab{D}{23}{24} ]
                                                                        [.\Lab{employee}{23}{25} ]
                                                                    ]
                                                                ]
                                                            ]	
                                                        ]												
                                                    ]
                                                ]
                                            ]
                                        ]
                                        [.\IBLab{V$'$}{14}{32}
                                            [.\Lab{stole}{32}{33} ]
                                            [.\Lab{office supplies}{32}{34} ]
                                        ]
                                    ]
                                ]
                            ]
						]
					]
				]
				[.\IBLab{V$'$}{5}{36}
					[.\Lab{worried}{36}{37} ]
                    [.\Lab{DP}{36}{38}
                        [.\Lab{the}{38}{39} ]
                        [.\Lab{executive}{38}{40} ]
                    ]
				]
			]
		]
	]
]

\draw[dashed,bend left] (TP1Subj) to (TP1);
\draw[dashed,bend left] (TP2Subj) to (TP2);
\draw[dashed,bend left=60] (TP3Subj) to (TP3);
\draw[dashed,bend right=30] (RCsource.north) to (RCtarget);
\end{tikzpicture}
\caption{Move-eager parse of sentential complement with embedded relative clause;
    \MaxTen$=31/31$,
    \MaxLex$=31/8$,
    \BoxTen$=8/5$,
    \BoxLex$=5/2$}
\label{fig:StablerParser_Der-SC-RC_Move-Eager}
\end{figure}

% RC/SC
\begin{figure}[tbph]
\TreeSize
\centering
%
\begin{tikzpicture}[
    scale=\TreeScale,
    level 1+/.style={level distance=\TreeDistanceValue},
    level 1/.style={sibling distance=-5em},
    level 2+/.style={sibling distance=-.5em},
    level 3/.style={sibling distance=-10em},
    level 4/.style={sibling distance=-20em},
    level 9/.style={sibling distance=-10em},
    level 10/.style={sibling distance=-20em}
    ]
\Tree
[.\Lab{CP}{0}{1}
	[.\Lab{C}{1}{2} ]
	[.\node(TP1){\Lab{TP}{1}{3}};
		[.\Lab{T$'$}{3}{4}
			[.\BLab{T}{4}{36} ]
			[.\Lab{VP}{4}{5}
				[.\node(TP1Subj){\Lab{DP}{5}{6}};
					[.\Lab{the}{6}{7} ]
					[.\node(RCtarget){\Lab{NP}{6}{8}};
						[.\Lab{N$'$}{8}{9}
							[.\BLab{who}{9}{17} ]
							[.\node(TP2){\Lab{TP}{9}{10}};
								[.\Lab{T$'$}{10}{11}
									[.\BLab{T}{11}{34} ]
									[.\Lab{VP}{11}{12}
										[.\node(TP2Subj){\IBLab{DP}{12}{18}};
											[.\Lab{the}{18}{19} ]
											[.\Lab{NP}{18}{20}
												[.\Lab{fact}{20}{21} ]
												[.\Lab{CP}{20}{22}
													[.\Lab{that}{22}{23} ]
													[.\node(TP3){\Lab{TP}{22}{24}};
														[.\Lab{T$'$}{24}{25}
															[.\BLab{T}{25}{30} ]
															[.\Lab{VP}{25}{26}
                                                                [.\node(TP3Subj){\Lab{DP}{26}{27}};
																	[.\Lab{the}{27}{28} ]
																	[.\Lab{employee}{27}{29} ]
																]
																[.\IBLab{VP}{26}{31}
																	[.\Lab{stole}{31}{32} ]
																	[.\Lab{office supplies}{31}{33} ]
																]
															]
														]
													]
												]											
											]				
										]
										[.\Lab{V$'$}{12}{13}
											[.\BLab{worried}{13}{35} ]
                                            [.\node(RCsource){\Lab{DP}{13}{14}};
                                                \Lab{D}{14}{15}
                                                \Lab{executive}{14}{16}
                                            ]
										]
									]
								]
							]
						]
					]
				]
				[.\IBLab{V$'$}{5}{37}
					[.\Lab{hired}{37}{38} ]
                    [.\Lab{DP}{37}{39}
                        [.\Lab{the}{39}{40} ]
                        [.\Lab{manager}{39}{41} ]
                    ]
				]
			]
		]
	]
]

\draw[dashed,bend left] (TP1Subj) to (TP1);
\draw[dashed,bend left] (TP2Subj) to (TP2);
\draw[dashed,bend left=60] (TP3Subj) to (TP3);
\draw[dashed,bend right] (RCsource.north) to (RCtarget);
\end{tikzpicture}
\caption{Move-eager parse of relative clause containing a sentential complement;
    \MaxTen$=32/32$,
    \MaxLex$=32/22$,
    \BoxTen$=8/5$,
    \BoxLex$=5/2$}
\label{fig:StablerParser_Der-RC-SC_Move-Eager}
\end{figure}

As can be seen in Tab.~\ref{tab:StablerParser_SC-RC-Table}, the Merge-eager and the Move-eager parsers behave almost exactly the same.
With both of them \MaxTen makes barely a difference between the two structures, whereas \MaxLex correctly prefers SC\slash RC if empty heads are ignored.
Results are mixed for the payload metrics.
With a Merge-eager parser they indeed capture the preference for SC\slash RC, but the Move-eager parser shows no difference for these metrics.
This isn't too surprising considering that payload wasn't a useful metric for the CFG parsing models either.
The surprising thing, then, is that the payload metrics actually work for Merge-eager parser.
%
\begin{table}[tbph]
    \centering
    \begin{subtable}[b]{0.475\textwidth}
        \centering
        \begin{tabular}{lcc}
                     & \textbf{SC\slash RC} & \textbf{RC\slash SC}\\\hline
             \BoxTen & 9/6                  & 14/11\\
             \BoxLex & 7/4                  & 12/9\\
             \MaxTen & 32/32                & 33/33\\
             \MaxLex & 32/9                 & 33/17\\
        \end{tabular}
    \caption{Merge-eager}
    \end{subtable}
    %
    \begin{subtable}[b]{0.475\textwidth}
        \centering
        \begin{tabular}{lcc}
                     & \textbf{SC\slash RC} & \textbf{RC\slash SC}\\\hline
             \BoxTen & 8/5                  & 8/5\\
             \BoxLex & 5/2                  & 5/2\\
             \MaxTen & 31/31                & 32/32\\
             \MaxLex & 31/8                 & 33/22\\
        \end{tabular}
    \caption{Move-eager}
    \end{subtable}
\caption{Overview of processing predictions for SC\slash RC and RC\slash SC}
\label{tab:StablerParser_SC-RC-Table}
\end{table}


\subsection{Subject Gaps and Object Gaps}
The results for subject gaps versus object gaps are slightly different.
Once again \MaxLex makes the right predictions for both parsers, and so do \BoxTen and \BoxLex (although the difference is much more pronounced with the Merge-eager parser in this case).
\MaxTen seems to predict a tie, but earlier in the course we already said that if two trees have the same \MaxTen value, we can rank them according to the second-highest tenure value.
With both parsers these values are 7/7 for the subject gap sentence and 10/9 for the object gap sentence, so a preference for subject gaps emerges even with \MaxTen.
%
\begin{table}[tbph]
    \centering
    \begin{subtable}[b]{0.475\textwidth}
        \centering
        \begin{tabular}{lcc}
                     & \textbf{SubjRC} & \textbf{ObjRC}\\\hline
             \BoxTen & 5/3             & 7/5\\
             \BoxLex & 3/1             & 6/4\\
             \MaxTen & 19/19           & 19/19\\
             \MaxLex & 19/7            & 19/9\\
        \end{tabular}
    \caption{Merge-eager}
    \end{subtable}
    %
    \begin{subtable}[b]{0.475\textwidth}
        \centering
        \begin{tabular}{lcc}
                     & \textbf{SubjRC} & \textbf{ObjRC}\\\hline
             \BoxTen & 5/3                  & 6/4\\
             \BoxLex & 3/1                  & 4/2\\
             \MaxTen & 19/19                & 19/19\\
             \MaxLex & 19/7                 & 19/9\\
        \end{tabular}
    \caption{Move-eager}
    \end{subtable}
\caption{Overview of processing predictions for SC\slash RC and RC\slash SC}
\label{tab:StablerParser_SubjObj-Table}
\end{table}

% SubjRC
\begin{figure}[tbph]
\TreeSize
\centering
%
\begin{tikzpicture}[
    scale=\TreeScale,
    level 1+/.style={level distance=\TreeDistanceValue},
    level 4/.style={sibling distance=-13em}
    ]
\Tree
[.\Lab{CP}{0}{1}
	[.\Lab{C}{1}{2} ]
	[.\node(TP1){\Lab{TP}{1}{3}};
		[.\Lab{T$'$}{3}{4}
			[.\BLab{T}{4}{23} ]
			[.\Lab{VP}{4}{5}
				[.\node(TP1Subj){\Lab{DP}{5}{6}};
					[.\Lab{the}{6}{7} ]
					[.\node(RCtarget){\Lab{NP}{6}{8}};
						[.\Lab{N$'$}{8}{9}
							[.\BLab{who}{9}{16} ]
							[.\node(TP2){\Lab{TP}{9}{10}};
								[.\Lab{T$'$}{10}{11}
									[.\BLab{T}{11}{17} ]
									[.\Lab{VP}{11}{12}
                                        [.\node(RCsource){\Lab{r}{12}{13}};
                                            \Lab{D}{13}{14}
                                            \Lab{reporter}{13}{15}
                                        ]
										[.\IBLab{V$'$}{12}{18}
											[.\Lab{attacked}{18}{19} ]
                                            [.\Lab{DP}{18}{20}
                                                [.\Lab{the}{20}{21} ]
                                                [.\Lab{senator}{20}{22} ]
                                            ]
										]
									]
								]
							]
						]
					]
				]
				[.\IBLab{V$'$}{5}{24}
					[.\Lab{admitted}{24}{25} ]
                    [.\Lab{DP}{24}{26}
                        [.\Lab{the}{26}{27} ]
                        [.\Lab{error}{26}{28} ]
                    ]
				]
			]
		]
	]												
]

\draw[dashed,bend left] (TP1Subj) to (TP1);
\draw[dashed,bend left=50] (RCsource) to (TP2);
\draw[dashed,bend left=30] (RCsource) to (RCtarget.south west);
\end{tikzpicture}
\caption{Merge-eager and Move-eager parse of relative clause with subject gap; 
    \MaxTen$=19/19$,
    \MaxLex$=19/7$,
    \BoxTen$=5/3$,
    \BoxLex$=3/1$}
\label{fig:StablerParser_DerSubjRC}
\end{figure}

% ObjRC
\begin{figure}[tbph]
\TreeSize
\centering
%
\begin{tikzpicture}[
    scale=\TreeScale,
    level 1+/.style={level distance=\TreeDistanceValue},
    level 4/.style={sibling distance=-20em}
    ]
\Tree
[.\Lab{CP}{0}{1}
	[.\Lab{C}{1}{2} ]
	[.\node(TP1){\Lab{TP}{1}{3}};
		[.\Lab{T$'$}{3}{4}
			[.\BLab{T}{4}{23} ]
			[.\Lab{VP}{4}{5}
				[.\node(TP1Subj){\Lab{DP}{5}{6}};
					[.\Lab{the}{6}{7} ]
					[.\node(RCtarget){\Lab{NP}{6}{8}};
						[.\Lab{N$'$}{8}{9}
							[.\BLab{who}{9}{18} ]
							[.\node(TP2){\Lab{TP}{9}{10}};
								[.\Lab{T$'$}{10}{11}
									[.\BLab{T}{11}{21} ]
									[.\Lab{VP}{11}{12}
                                        [.\node(TP2Subj){\Lab{DP}{12}{13}};
                                            [.\BLab{the}{13}{19} ]
                                            [.\BLab{senator}{13}{20} ]
                                        ]
										[.\Lab{V$'$}{12}{14}
											[.\BLab{attacked}{14}{22} ]
											[.\node(RCsource){\Lab{DP}{14}{15}};
                                                \Lab{D}{15}{16}
                                                \Lab{reporter}{15}{17}
                                            ]
										]
									]
								]
							]
						]
					]
				]
				[.\IBLab{V$'$}{5}{24}
					[.\Lab{admitted}{24}{25} ]
                    [.\Lab{DP}{24}{26}
                        [.\Lab{the}{26}{27} ]
                        [.\Lab{error}{26}{27} ]
                    ]
				]
			]
		]
	]
]

\draw[dashed,bend left=45] (TP1Subj) to (TP1);
\draw[dashed,bend left=45] (TP2Subj) to (TP2);
\draw[dashed,bend right] (RCsource) to (RCtarget);
\end{tikzpicture}
\caption{Merge-eager parse of relative clause with object gap;
    \MaxTen$=19/19$,
    \MaxLex$=19/9$,
    \BoxTen$=7/5$,
    \BoxLex$=6/4$}
\label{fig:StablerParser_Der-ObjRC_Merge-Eager}
\end{figure}

% ObjRC
\begin{figure}[tbph]
\TreeSize
\centering
%
\begin{tikzpicture}[
    scale=\TreeScale,
    level 1+/.style={level distance=\TreeDistanceValue},
    level 4/.style={sibling distance=-20em}
    ]
\Tree
[.\Lab{CP}{0}{1}
	[.\Lab{C}{1}{2} ]
	[.\node(TP1){\Lab{TP}{1}{3}};
		[.\Lab{T$'$}{3}{4}
			[.\BLab{T}{4}{23} ]
			[.\Lab{VP}{4}{5}
				[.\node(TP1Subj){\Lab{DP}{5}{6}};
					[.\Lab{the}{6}{7} ]
					[.\node(RCtarget){\Lab{NP}{6}{8}};
						[.\Lab{N$'$}{8}{9}
							[.\BLab{who}{9}{17} ]
							[.\node(TP2){\Lab{TP}{9}{10}};
								[.\Lab{T$'$}{10}{11}
									[.\BLab{T}{11}{21} ]
									[.\Lab{VP}{11}{12}
                                        [.\node(TP2Subj){\IBLab{DP}{12}{18}};
                                            [.\Lab{the}{18}{19} ]
                                            [.\Lab{senator}{18}{20} ]
                                        ]
										[.\Lab{V$'$}{12}{13}
											[.\BLab{attacked}{13}{22} ]
											[.\node(RCsource){\Lab{DP}{13}{14}};
                                                \Lab{D}{14}{15}
                                                \Lab{reporter}{14}{16}
                                            ]
										]
									]
								]
							]
						]
					]
				]
				[.\IBLab{V$'$}{5}{24}
					[.\Lab{admitted}{24}{25} ]
                    [.\Lab{DP}{24}{26}
                        [.\Lab{the}{26}{27} ]
                        [.\Lab{error}{26}{27} ]
                    ]
				]
			]
		]
	]
]

\draw[dashed,bend left=45] (TP1Subj) to (TP1);
\draw[dashed,bend left=45] (TP2Subj) to (TP2);
\draw[dashed,bend right] (RCsource) to (RCtarget);
\end{tikzpicture}
\caption{Move-eager parse of relative clause with object gap; 
    \MaxTen$=19/19$,
    \MaxLex$=19/9$,
    \BoxTen$=6/4$,
    \BoxLex$=4/2$}
\label{fig:StablerParser_DerObjRC_Move-Eager}
\end{figure}

\bibliographystyle{../../linquiry3}
\bibliography{../../universal,../../graf}
