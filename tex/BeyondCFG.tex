\chapter{Parsing Beyond Context-Free Grammars}
\label{cha:BeyondCFG}

\section{Not All Natural Languages are Context-Free}

\subsection{Context-Free Pumping Lemma}

\begin{theorem}[CFL Pumping Lemma]
    If $L$ is a context-free string language, then there is some constant $k \geq 0$ such that
    %
    \begin{itemize}
        \item $uxvyw \in L$, and
        \item $|uxvyw| \geq k$
    \end{itemize}
    %
    jointly imply $u x^n v y^n w \in L$ for all $n \geq 1$.
\end{theorem}
%
tree loop
%
\begin{lemma}
    None of the following languages are context-free:
    %
    \begin{itemize}
        \item $a_1^n a_2^n \cdots a_i^n$, for any fixed choice of $i \geq 3$.
        \item $\setof{ww \mid w \in \Sigma^*}$,
        \item $a^{2^n}$
    \end{itemize}
\end{lemma}

\subsection{Mildly Context-Sensitive Languages}

\begin{definition}[Mild Context-Sensitivity]
    A class of string languages is \emph{mildly context-sensitive} iff
    %
    \begin{enumerate}
        \item it properly includes the class of context-free languages,
        \item each language can be generated by a grammar that allows for parsing in polynomial time,
        \item each language is of constant growth,
        \item each language is semilinear.
    \end{enumerate}
\end{definition}

\section{Parikh's Theorem}

\subsection{Semilinear Languages}

\begin{definition}[Vector operations]
    Given two vectors $\vec{u}, \vec{v} \in \NatNum^k$, their sum $\vec{u} + \vec{v}$ is the vector $\tuple{u_1 + v_1, u_2 +v_2, \ldots, u_k+v_k}$.
\end{definition}

\begin{definition}[Parikh Map]
    The \emph{Parikh vector} of a string is the function $\ParikhVec: \Sigma^* \rightarrow \NatNum^{\cardof{\Sigma}}$ such that
    %
    \[
        \ParikhVec(a_1 a_2 \cdots a_n) \is
            \begin{cases}
                0^{i-1} \cdot \tuple{1} \cdot 0^{k-i} & \text{ if $n = 1$ and $a_1$ is the $i$-th symbol of $\Sigma$,}\\
                \ParikhVec(a_1) + \ParikhVec(a_2 \cdots a_n) & \text{otherwise}
            \end{cases}
    \]
    The \emph{Parikh map} $\ParikhMap$ associates a language $L$ with the set $\setof{ \ParikhVec(w) \mid w \in L}$.
\end{definition}

\begin{definition}[Semilinearity]
    A set $S \subseteq \NatNum^k$ is \emph{linear} iff it is of the form $\{u_0 + t_1 u_1 + \cdots + t_n u_n | t_1, \ldots, t_n \in \NatNum\}$ for some fixed vectors $u_0, u_1, \ldots, u_n$, $n \geq 0$.
    A set is \emph{semilinear} iff it is the union of a finite number of linear sets.
    A language is semilinear iff its image under the Parikh map is semilinear.
\end{definition}

\subsection{Semilinearity and Constant Growth}

\begin{theorem}
    Every semilinear set is of constant-growth.
\end{theorem}

\begin{corollary}
    $a^{2^n}$ is not semilinear.
\end{corollary}

\begin{exercise}
    Give an example of a language that is of constant-growth but not semilinear.
\end{exercise}

\subsection{Semilinearity = Regular Backbone + String Permutation}

\begin{theorem}[Parikh's Theorem]
    For every semilinear language $L$ there is a regular language $R$ such that $\ParikhMap(L) = \ParikhMap(R)$.
\end{theorem}
%
\begin{exercise}
    The inverse of Parikh's Theorem also holds: for every regular language $R$ there is a semilinear language $L$ such that $\ParikhMap(L) = \ParikhMap(R)$.
    Explain why this is (trivially) the case.
\end{exercise}

$\ParikhMap(a_1^n a_2^n \cdots a_i^n) = \ParikhMap(a_1 a_2 \cdots a_i)^n$

\section{Minimalist Grammars}

\section{A CKY Parser for Minimalist Grammars}
