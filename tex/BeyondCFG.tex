\chapter{Moving Beyond Context-Free Grammars}
\label{cha:BeyondCFG}

\section{Not All Natural Languages are Context-Free}

\subsection{Context-Free Pumping Lemma}

\begin{theorem}[CFL Pumping Lemma]
    If $L$ is a context-free string language, then there is some constant $k \geq 0$ such that
    %
    \begin{itemize}
        \item $\String{uxvyw} \in L$, and
        \item $|\String{uxvyw}| \geq k$
    \end{itemize}
    %
    jointly imply $\String{u x^n v y^n w} \in L$ for all $n \geq 1$.
\end{theorem}
%
\begin{center}
    \begin{tikzpicture}[%
        every node/.style = {minimum size = 0em, inner sep = 0em}]

        \node (0) at (0,0) {};
        \node[xshift=15em,yshift=2em] (1) at (0) {};

        \foreach \Prefix in {0,01,011,1,11,111,1111}
            {
            \node (\Prefix0) [below left=5em of \Prefix] {};
            \node (\Prefix2) [below right=5em of \Prefix] {};
            \node (\Prefix1) at ($(\Prefix0) !.5! (\Prefix2)$) {};
            }

        \foreach \Prefix/\Color in {%
                0/SeaGreen4,
                01/blue!75,
                011/purple!70,
                1/SeaGreen4,
                11/blue!75,
                111/blue!75,
                1111/purple!70%
            }
            \draw[draw=\Color, fill=\Color] (\Prefix.center) -- (\Prefix0.center) -- (\Prefix2.center) -- cycle;

        \foreach \String/\Start/\End in {%
            u/00/01,
            u/10/11,
            x/010/011,
            x/110/111,
            x/1110/1111,
            v/0110/0112,
            v/11110/11112,
            y/011/012,
            y/111/112,
            y/1111/1112,
            w/01/02,
            w/11/12%
            }
            \node[yshift=-.5em] at ($(\Start) !.5! (\End)$) {$\String$};

        \node[xshift=-7.5em] at (111) {$\Rightarrow$};
    \end{tikzpicture}
\end{center}
%
\begin{lemma}
    None of the following languages are context-free:
    %
    \begin{itemize}
        \item $a_1^n a_2^n \cdots a_i^n$, for any fixed choice of $i \geq 3$.
        \item $\setof{ww \mid w \in \Sigma^*, \cardof{\Sigma} \geq 2}$,
        \item $a^{2^n}$
    \end{itemize}
\end{lemma}

\subsection{Mildly Context-Sensitive Languages}

\begin{definition}[Mild Context-Sensitivity]
    A class of string languages is \emph{mildly context-sensitive} iff
    %
    \begin{enumerate}
        \item it properly includes the class of context-free languages,
        \item each language can be generated by a grammar that allows for parsing in polynomial time,
        \item each language is of constant growth,
        \item each language can be generated by a grammar with only a bounded number of crossing dependencies.
    \end{enumerate}
\end{definition}
%
\begin{exercise}
    Show that $a^n b^n c^n$ is of constant growth, whereas $a^{2^n}$ is not.
\end{exercise}
%
This definition is not particularly elegant as it puts restrictions on the languages as well as the grammars that generate them.
A more unified definition in terms of just one of the two would be more appealing.
In addition, the clause about crossing dependencies is very vague.
Fortunately it can be made more precise via the concept of \emph{semilinearity}.

\subsection{Semilinear Languages}

In order to define semilinearity, we first need to introduce the \emph{Parikh map}.
Intuitively, the Parikh map counts for each symbol of the alphabet how often it occurs in a given string.
This is modeled mathematically as a function that takes a string as input and returns a vector of integers --- the \emph{Parikh vector} --- such that the $i$th integer is the number of occurrences of the $i$th symbol of the alphabet (so we have to posit some arbitrary linear order for our alphabet symbols).

\begin{definition}[Vector addition]
    Let $\vec{u} \is \tuple{u_1, \ldots, u_k}$ and $\vec{v} \is \tuple{v_1, \ldots, v_k}$ be vectors in $\NatNum^k$.
    Then their sum $\vec{u} + \vec{v}$ is the vector $\tuple{u_1 + v_1, u_2 +v_2, \ldots, u_k+v_k}$.
\end{definition}

\begin{definition}[Parikh Map]
    The \emph{Parikh vector} of a string is the function $\ParikhVec: \Sigma^* \rightarrow \NatNum^{\cardof{\Sigma}}$ such that
    %
    \[
        \ParikhVec(a_1 a_2 \cdots a_n) \is
            \begin{cases}
                0^{i-1} \cdot \tuple{1} \cdot 0^{k-i} & \text{ if $n = 1$ and $a_1$ is the $i$-th symbol of $\Sigma$,}\\
                \ParikhVec(a_1) + \ParikhVec(a_2 \cdots a_n) & \text{otherwise}
            \end{cases}
    \]
    The \emph{Parikh map} $\ParikhMap$ associates a language $L$ with the set $\setof{ \ParikhVec(w) \mid w \in L}$.
    Two languages $L$ and $L'$ are \emph{letter equivalent} iff $\ParikhMap(L) = \ParikhMap(L')$.
\end{definition}

\begin{examplebox}[Parikh vectors and maps]
    Consider the string language $\setof{a,b}^*$, which contains all possible strings over $a$ and $b$.
    The Parikh map computes the following Parikh vectors for this language.
    %
    \begin{center}
        \begin{tabular}{rc}
            \textbf{String} & \textbf{Parikh Vector}\\
            $\emptystring$ & $\tuple{0,0}$\\
            $\String{a}$ & $\tuple{1,0}$\\
            $\String{b}$ & $\tuple{0,1}$\\
            $\String{aa}$ & $\tuple{2,0}$\\
            $\String{ab}$ & $\tuple{1,1}$\\
            $\String{ba}$ & $\tuple{1,1}$\\
            $\String{bb}$ & $\tuple{0,2}$\\
            $\vdots$ & $\vdots$
        \end{tabular}
    \end{center}
    %
    The set of all Parikh vectors is $\setof{ \tuple{m,n} \mid m,n \in \NatNum }$, which is identical to $\NatNum \times \NatNum$.
    In a certain sense, then, we can think of $\setof{a,b}^*$ as the set of all vectors in the first quadrant of the Cartesian plane.
    This is a simplification, though, since $\setof{a,b}^*$ contains many strings that map to the same Parikh vector because the Parikh map completely ignores linear order.
\end{examplebox}
%
\begin{exercise}
    Define a string language that is distinct from $\setof{a,b}^*$ but letter-equivalent to it.
\end{exercise}
%
\begin{exercise}
    Define a string language as in the previous exercise, with the additional requirement that no proper subset of that language is letter equivalent to $\setof{a,b}^*$. 
\end{exercise}

The set $\NatNum \times \NatNum$ is obviously infinite, but nonetheless it can be generated from a finite number of vectors.
Any vector of that set is a directed arrow in the first quadrant of that Cartesian plane.
Wherever that vector points, we can get to that point by only moving up and to the right.
For example, the location pointed to by the vector $(2,1)$ can be reached by starting at $(0,0)$ and taking two steps to the right, immediately followed by one step up.
This can be written as $(2,1) = (0,0) + (1,0) + (1,0) + (0,1) = (0,0) + 2 \cdot (1,0) + 1 \cdot (0,1)$.
Every vector of $\NatNum \times \NatNum$ can be described in this fashion, we just have to change the multipliers accordingly.
Consequently, $\NatNum \times \NatNum$ is equivalent to the set $\setof{(0,0) + t_1 \cdot (1,0) + t_2 \cdot (0,1) \mid t_1, t_2 \in \NatNum}$.
Sets of this form are called \emph{linear}.
%
\begin{definition}[Linear Set]
    A set $S \subseteq \NatNum^k$ is \emph{linear} iff it is of the form $\{u_0 + t_1 u_1 + \cdots + t_n u_n | t_1, \ldots, t_n \in \NatNum\}$ for some fixed base vectors $u_0, u_1, \ldots, u_n$, $n \geq 0$.
\end{definition}
%
\begin{exercise}
    Show that the Parikh map yields a linear set for $a^n b^n c^n$, $n \geq 1$.
\end{exercise}

Many interesting languages are not linear under the Parikh map.
However, they are still semilinear

\begin{definition}[Semilinearity]
    A set is \emph{semilinear} iff it is the union of a finite number of linear sets.
    A language is semilinear iff its image under the Parikh map is semilinear.
\end{definition}

\subsection{Semilinearity and Constant Growth}

\begin{theorem}
    Every semilinear set is of constant-growth.
\end{theorem}

\begin{corollary}
    $a^{2^n}$ is not semilinear.
\end{corollary}
%
\begin{exercise}
    Give an example of a language that is of constant-growth but not semilinear.
\end{exercise}

\subsection{Semilinearity = Regular Backbone + String Permutation}

\begin{theorem}[Parikh's Theorem]
    For every semilinear language $L$ there is a regular language $R$ such that $\ParikhMap(L) = \ParikhMap(R)$.
\end{theorem}
%
\begin{exercise}
    The inverse of Parikh's Theorem also holds: for every regular language $R$ there is a semilinear language $L$ such that $\ParikhMap(L) = \ParikhMap(R)$.
    Explain why this is (trivially) the case.
\end{exercise}


\section{Minimalist Grammars}
