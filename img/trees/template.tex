% Template to be used at writelatex.com
%
% Workflow
%  - edit tikzpicture to produce annotated parse tree
%  - check output via writelatex
%  - copy paste all the code from \begin{tikzpicture} to \end{tikzpicture} into a new file
%  - add your full name as a comment at the top of the file
%  - add description of parse as comment to the top of the file (mention at least the parameters below)
%  - name file according to following template
%     LANG_RC_ANALYSIS_MOVE_LCA.tex
%    where
%         LANG in {eng, jap, chi, kor} % what language?
%         RC in {src, orc} % what type of relative clause?
%         ANALYSIS in {prom, wh, bind} % what type of relative clause analysis? promotion, wh-movement, or just bound pro without movement?
%         MOVE in {head,nohead} % do we indicate head movement?
%         LCA in {lca,nolca} % are the trees antisymmetric?
%  - upload file to tex folder in Google drive

%%%%%%%%%%%%%%
%  Preamble  %
%%%%%%%%%%%%%%
\documentclass[12pt]{article}

\usepackage{fixltx2e}
\usepackage[utf8]{inputenc}
\usepackage[T1]{fontenc}
\usepackage[a3paper,margin=.75cm]{geometry}
\usepackage{amsmath}
\usepackage{tikz-qtree}

% you don't need to specify features in the derivation tree,
% but if you think it's necessary for some reason, use the MG commands below

% MG feature types
\newcommand{\FeatFont}[1]{\ensuremath{\mathrm{#1}}} % typeset features in math roman
\newcommand{\fcat}[1]{\ensuremath{\FeatFont{#1}}\ } % category feature
\newcommand{\fsel}[1]{\ensuremath{\mathop{=}\FeatFont{#1}\ }} % selector feature
\newcommand{\flcr}[1]{\ensuremath{\mathop{+}\FeatFont{#1}\ }} % licensor feature
\newcommand{\flce}[1]{\ensuremath{\mathop{-}\FeatFont{#1}\ }} % licensee feature

% MG lexical item
% Format:
%	argument 1: phonetic exponent
%	argument 2: feature string (use feature macros)
% Example:
%	an empty C-head: \mlex{\emptystring}{\fsel{T} \fcat{C}}
%   wh-moving which: \mlex{which}{\fsel{N} \fcat{D} \flce{wh}}
\newcommand{\mlex}[2]{\ensuremath{\text{#1} ::\thinspace #2}}

% empty string
\newcommand{\emptystring}{\ensuremath{\varepsilon}}

% Labeling Macros
% Format:
%	argument 1 = node label
%	argument 2 = step at which node is put into memory
% 	argument 3 = step at which node is flushed from memory
% Example:
%	Merge node, not held in memory: \Lab{Merge}{5}{6}
%   lexical item, held in memory: \BLab{the}{2}{15}
%   interior node, held in memory: \IBLab{TP}{2}{15}
\newcommand{\Lab}[3]{\textsuperscript{#2}#1\textsubscript{#3}} % for normal nodes
\newcommand{\BLab}[3]{\textsuperscript{#2}#1\boxed{\textsubscript{#3}}} % for boxed leaves
\newcommand{\IBLab}[3]{\textsuperscript{#2}#1\boxed{\textsubscript{#3}}} % for boxed interior nodes

% Derivation Tree Typesetting
\newcommand{\TreeSize}{}   % set font size for node labels
\newcommand{\TreeScale}{1} % set scaling factor of entire image
\newcommand{\TreeDistanceValue}{2.5em} % vertical distance between nodes


%%%%%%%%%%%%%
%  Content  %
%%%%%%%%%%%%%

\begin{document}

% annotated MG derivation
% tikz-qtree primer:
%    - trees are written as labeled bracketing via syntax [.LABEL subtree1 subtree2 ... ]
%    - for LABEL, use one of the tree labeling macros from the preamble
%    - for bar levels, the ' must be surrounded by $, i.e. $'$ (otherwise we get an apostrophe)
%    - brackets must not touch (something like ]] will give compilation errors)
%    - to add movement branches, you need to name nodes
%           [.\node(NameOfNode){NodeLabel};
%    - don't forget about the semicolon in this case
%    - use descriptive names!
%    - you can then draw movement branches with the draw command
%           \draw[move,DIRECTION] (source) to (target);
%      where DIRECTION in {bend left, bend right}
%    - the direction can be followed by =n, n in [0,360], which defines the degree of curving

\begin{tikzpicture}[
        scale = \TreeScale,
        move/.style = {dashed,blue},
        level 1+/.style = { level distance = \TreeDistanceValue }
    ]
    \Tree
        [.\Lab{CP}{0}{1}
            [.\Lab{C}{1}{2} ]
            [.\node(TP1){\Lab{TP}{1}{3}};
                [.\Lab{T$'$}{3}{4}
                    [.\BLab{T}{4}{23} ]
                    [.\Lab{VP}{4}{5}
                        [.\node(TP1Subj){\Lab{DP}{5}{6}};
                            [.\Lab{the}{6}{7} ]
                            [.\node(RCtarget){\Lab{NP}{6}{8}};
                                [.\Lab{N$'$}{8}{9}
                                    [.\BLab{who}{9}{16} ]
                                    [.\node(TP2){\Lab{TP}{9}{10}};
                                        [.\Lab{T$'$}{10}{11}
                                            [.\BLab{T}{11}{17} ]
                                            [.\Lab{VP}{11}{12}
                                                [.\node(RCsource){\Lab{r}{12}{13}};
                                                    [.\Lab{D}{13}{14} ]
                                                    [.\Lab{reporter}{13}{15} ]
                                                ]
                                                [.\IBLab{V$'$}{12}{18}
                                                    [.\Lab{attacked}{18}{19} ]
                                                    [.\Lab{DP}{18}{20}
                                                        [.\Lab{the}{20}{21} ]
                                                        [.\Lab{senator}{20}{22} ]
                                                    ]
                                                ]
                                            ]
                                        ]
                                    ]
                                ]
                            ]
                        ]
                        [.\IBLab{V$'$}{5}{24}
                            [.\Lab{admitted}{24}{25} ]
                            [.\Lab{DP}{24}{26}
                                [.\Lab{the}{26}{27} ]
                                [.\Lab{error}{26}{28} ]
                            ]
                        ]
                    ]
                ]
            ]												
        ]

    \draw[move,bend left] (TP1Subj) to (TP1);
    \draw[move,bend left=50] (RCsource) to (TP2);
    \draw[move,bend left=30] (RCsource) to (RCtarget.south west);

\end{tikzpicture}

\end{document}
