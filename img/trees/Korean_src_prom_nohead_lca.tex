% Gianpaul Rachiele
% Korean, Subject Relative Clause, promotion analysis, No head movement, LCA
\documentclass[12pt]{article}

\usepackage{fixltx2e}
\usepackage[utf8]{inputenc}
\usepackage[T1]{fontenc}
\usepackage[a3paper,margin=.75cm]{geometry}
\usepackage{amsmath}
\usepackage{tikz-qtree}

% you don't need to specify features in the derivation tree,
% but if you think it's necessary for some reason, use the MG commands below

% MG feature types
\newcommand{\FeatFont}[1]{\ensuremath{\mathrm{#1}}} % typeset features in math roman
\newcommand{\fcat}[1]{\ensuremath{\FeatFont{#1}}\ } % category feature
\newcommand{\fsel}[1]{\ensuremath{\mathop{=}\FeatFont{#1}\ }} % selector feature
\newcommand{\flcr}[1]{\ensuremath{\mathop{+}\FeatFont{#1}\ }} % licensor feature
\newcommand{\flce}[1]{\ensuremath{\mathop{-}\FeatFont{#1}\ }} % licensee feature

% MG lexical item
% Format:
%	argument 1: phonetic exponent
%	argument 2: feature string (use feature macros)
% Example:
%	an empty C-head: \mlex{\emptystring}{\fsel{T} \fcat{C}}
%   wh-moving which: \mlex{which}{\fsel{N} \fcat{D} \flce{wh}}
\newcommand{\mlex}[2]{\ensuremath{\text{#1} ::\thinspace #2}}

% empty string
\newcommand{\emptystring}{\ensuremath{\varepsilon}}

% Labeling Macros
% Format:
%	argument 1 = node label
%	argument 2 = step at which node is put into memory
% 	argument 3 = step at which node is flushed from memory
% Example:
%	Merge node, not held in memory: \Lab{Merge}{5}{6}
%   lexical item, held in memory: \BLab{the}{2}{15}
%   interior node, held in memory: \IBLab{TP}{2}{15}
\newcommand{\Lab}[3]{\textsuperscript{#2}#1\textsubscript{#3}} % for normal nodes
\newcommand{\BLab}[3]{\textsuperscript{#2}#1\boxed{\textsubscript{#3}}} % for boxed leaves
\newcommand{\IBLab}[3]{\textsuperscript{#2}#1\boxed{\textsubscript{#3}}} % for boxed interior nodes

% Derivation Tree Typesetting
\newcommand{\TreeSize}{}   % set font size for node labels
\newcommand{\TreeScale}{1} % set scaling factor of entire image
\newcommand{\TreeDistanceValue}{2.5em} % vertical distance between nodes


%%%%%%%%%%%%%
%  Content  %
%%%%%%%%%%%%%

\begin{document}

% annotated MG derivation
% tikz-qtree primer:
%    - trees are written as labeled bracketing via syntax [.LABEL subtree1 subtree2 ... ]
%    - for LABEL, use one of the tree labeling macros from the preamble
%    - for bar levels, the ' must be surrounded by $, i.e. $'$ (otherwise we get an apostrophe)
%    - brackets must not touch (something like ]] will give compilation errors)
%    - to add movement branches, you need to name nodes
%           [.\node(NameOfNode){NodeLabel};
%    - don't forget about the semicolon in this case
%    - use descriptive names!
%    - you can then draw movement branches with the draw command
%           \draw[move,DIRECTION] (source) to (target);
%      where DIRECTION in {bend left, bend right}
%    - the direction can be followed by =n, n in [0,360], which defines the degree of curving

\begin{tikzpicture}[
        scale = \TreeScale,
        move/.style = {dashed,blue},
        level 1/.style = { level distance = \TreeDistanceValue }
    ]
    \Tree
        [.\Lab{CP}{1}{2}
            [.\Lab{C}{2}{3} ]
            [.\node(TP1Spec){\Lab{TP}{2}{4}};
                [.\Lab{T$'$}{4}{5}
                    [.\BLab{T}{5}{27} ]
                    [.\node(vP1Spec){\Lab{vP}{5}{6}};
                    		[.\Lab{vP}{6}{7}
                        [.\node(Subject){\Lab{DP}{7}{8}};
                        [.\Lab{D}{8}{9} ]
                            [.\node(TP2land){\Lab{RelP}{8}{10}};
                                [.\Lab{Rel$'$}{10}{11}
                                    [.\BLab{-n}{11}{24} ]
                                    [.\node(CP2Spec){\Lab{CP}{11}{12}};
                                        [.\Lab{C$'$}{12}{13}
                                            [.\BLab{C}{13}{26} ]
                                            [.\node(TP2Spec){\Lab{TP}{13}{14}};
                                                [.\Lab{T$'$}{14}{15}
                                                    [.\BLab{T}{15}{16} ]
                                                    [.\node(vP2Spec){\Lab{vP}{15}{17}};
                                                    [.\Lab{vP}{17}{18}
                                                        [.\node(headNP){\Lab{tycoon}{18}{25}}; ]
                                                        [.\IBLab{v$'$}{18}{19}
                                                        [.\IBLab{v}{19}{22} ]
                                                        [.\Lab{VP}{19}{20}
                                                            [.\Lab{invited}{20}{23} ]
                                                            [.\node(headNP2){\BLab{mayor}{20}{21}}; ]
                                                        ]
                                                    ]
                                                ]
                                            ]
                                        ]
                                    ]
                                ]
                            ]
                        ]
                    ]
                ]
                        [.\IBLab{v$'$}{7}{28}
                        	[.\IBLab{v}{28}{31} ] 
                            [.\Lab{VP}{28}{29} 
                            	[.\IBLab{loves}{29}{32} ]
                            	[.\node(headNP3){\Lab{money}{29}{30}}; ] 
                        ]
                    ]
                ]
            ]												
        ]
    ]
]

    \draw[move,bend left=100] (Subject) to (TP1Spec);
    \draw[move,bend left=40] (headNP) to (TP2Spec);
    \draw[move,bend left=40] (TP2Spec) to (CP2Spec);
    \draw[move,bend right=100] (headNP2) to (vP2Spec);
    \draw[move,bend left=90] (TP2Spec) to (TP2land);

\end{tikzpicture}


Payload = 9/4\\
PayloadLex = 8/3\\
Tenure = 22/13\\
TenureLex = 22/13

\end{document}
